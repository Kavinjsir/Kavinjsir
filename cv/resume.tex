% !TEX program = xelatex

\documentclass{resume}

\begin{document}
\pagenumbering{gobble} % suppress displaying page number

\name{Tony Jin}

\basicInfo{
  \email{kavinjsir@gmail.com} \textperiodcentered\ 
  \phone{(+1)617-870-9580} \textperiodcentered\ 
  \github[Kavinjsir]{https://github.com/Kavinjsir} \textperiodcentered\
  \linkedin[Tony Jin]{https://www.linkedin.com/in/tony-j-84b223146/}}

\section{\faGraduationCap\ Education}
\datedsubsection{\textbf{Boston University (BU)}, Boston, United States}{2021 -- 2023}
\textit{Master of Science} in Computer Science (CS), concentrate on network
\datedsubsection{\textbf{Shanghai Jiao Tong University (SJTU)}, Shanghai, China}{2014 -- 2018}
\textit{Bachelor of Science} in Software Engineering (SE)

\section{\faCogs\ Skills}
\begin{itemize}[parsep=0.5ex]
  \item Passionate software engineer with rich backgrounds of SRE \& DevOps; Member of Kubernetes SIGs
  \item Infra Stacks: Kubernetes, AWS, Azure, Terraform, Nginx, Gitlab-CI, Azure DevOps
  \item Monitoring Stacks: Prometheus, Grafana, Thanos, Loki, Jaeger, OpenTelemetry
  \item Programming Languages: Go, TypeScript, Linux shell
  \item App Stacks: ReactJS, Express.JS, PostgreSQL
\end{itemize}

\section{\faUsers\ Working Experience}
\datedsubsection{\textbf{Microsoft} | Senior Software Engineer(Contractor), Azure Operator Nexus | Boston, MA}{Jan 2023 -- }
% \role{DevOps Engineer}{Infra Team}
Pioneered cloud infrastructure development to support Azure Operator Nexus products
\begin{itemize}
  \item Enhanced network infrastructure's reliability and efficiency using Kubernetes operators.
  \item Augmented testing automation via KubeVirt, ensuring stringent validation of Azure product criteria.
  \item Harmonized Azure DevOps pipelines, bolstering quality assurance and accelerating lifecycle efficiency.
  \item Stacks: Golang, Kubernetes, Linux Shell, Azure, Azure DevOps 
%  \item Empowered the network infrastructure with high reliability and efficiency by Kubernetes operators.
%  \item Enriched testing automation with virtual resources by Kubebirt to validate Azure products criteria.
%  \item Standardized Azure DevOps pipelines to achieve high quality assurance and boost lifecycle efficiency.
%  \item Stacks: Golang, Kubernetes, Linux Shell, Azure, Azure DevOps 
% \item Created standardized Azure DevOps pipeline Docker image to support common tooling such as Golang, Helm, and linters for multiple repositories across Azure for Operators organization
% \item Introduced Semantic Versioning for publishing Docker images for developers to better understand impact of changes
% \item Enhanced end-to-end testing CI/CD pipeline to automatically deploy virtual infrastructure to simulate bare metal environment using bash and reusable Azure DevOps pipeline templates Explored transforming other teams' repositories for general usage across the organization for common CI/CD pipeline scenarios
% \item Stacks: Golang, Kubernetes, Azure DevOps.
\end{itemize}

\datedsubsection{\textbf{Second Spectrum} | Software Engineer, Infra Team | Los Angeles, CA}{Feb 2018 -- Sep 2021}
% \role{DevOps Engineer}{Infra Team}
Oversaw various infrastructure platforms to streamline operations across the company
\begin{itemize}
%  \item Explored and standardized AWS workflows with IaC tools to safely and efficiently provision app resources.
%  \item Improved infrastructure to simplify operations on the apps' lifecycle and speeded up deployment process.
%  \item Implemented a telemetry system providing plentiful and precise metrics to efficiently track +200 apps.
%  \item Dynamically allocated resources with the integration of K8s Hpa and AWS Cluster Autoscaler.
  \item Optimized and standardized Infrastructure as Code workflows, ensuring reliable AWS provisioning.
  \item Enhanced infrastructure performance, simplifying app lifecycle and expediting the deployment process.
  \item Implemented an advanced telemetry system, offering a rich set of metrics covering 200 applications.
  \item Streamlined resource allocation through seamless integration of K8s HPA and AWS Cluster Autoscaler.
  \item Stacks: Golang, TypeScript, Kubernetes, AWS, Terraform, Prometheus, Grafana, Loki
\end{itemize}

% \datedsubsection{\textbf{Second Spectrum} | Software Engineer, App Team | Shanghai, China}{Jan 2018 -- Jan 2019}
% % \role{Full Stack Engineer}{App Team}
% Developed and Maintained SaaS Apps
% \begin{itemize}
%   \item Created a web service of video players to match NBA video chunks with moving 2D dots frames.
%   \item Created a dashboard to dynamically display real-time video transcoding status with ReactJS.
%   \item Evaluated and upgraded code with reliable dev-stack, improving code readability and maintainability.
%   \item Stacks: TypeScript, ReactJS, Webpack, AntDesign, Nginx, Koa
% \end{itemize}

\datedsubsection{\textbf{Morgan Stanley} | Software Engineer Intern, Fin Tech | Shanghai, China}{Jun 2017 -- Sep 2017}
% \role{Technology Intern}{Fin Tech}
Developed a Clearing System for the Sydney Futures Exchange (SFE)
\begin{itemize}
  \item Implemented upstream components in the SFE clearing system, ensuring superior security measures.
  \item Incorporated a message queue system to securely and flexibly transform and route data.
  \item Carried out comprehensive unit tests achieving 90\% coverage.
  \item Stacks: Apache Camel, ActiveMQ, Java.
\end{itemize}

\section{\faUsers\ Contribution Experience}

\datedsubsection{\textbf{The Kubebuilder Community} | Repository Reviewer, Contributor, Kubernetes-sigs}{Mar 2022 -- }
% \role{Developer}{}
Active Maintenance Contributor(Reviewer) within the Kubebuilder Community
\begin{itemize}
  \item Involved in the design and implementation of new features, enhancements to the codebase, CI/CD improvements, and proposal submissions.
  \item Regular contributions consist of participating in bi-weekly meetings, reviewing pull requests, engaging in community Q\&A, and maintaining up-to-date documentation.
  \item Achievements:
    \docs [KubeCon 2023 EU Speaker]{https://sched.co/1HyUw} \textperiodcentered\
    \docs [GSoC 2023 Mentor]{https://github.com/cncf/mentoring/pull/844} \textperiodcentered\
    \youtube[GSoC 2022 Mentee]{https://www.youtube.com/watch?v=-w\_JjcV8jXc&t=2s&ab\_channel=CamilaMacedo} \textperiodcentered\
  % \item Stacks: Go, Kubernetes
\end{itemize}

% \datedsubsection{\textbf{Google Summer of Code 2022} | Kubernetes Contributor, CNCF}{Jun 2022 -- Oct 2022}
% % \role{Developer}{}
% A Grafana Dashboard Plugin to Observe K8s Operator Status
% \begin{itemize}
%   \item Created a new plugin to scaffold Grafana manifests for operator status visualization.
%   \item Expanded functionalities to support scaffolding on user-defined custom metrics.
%   % \item Stacks: Go, Grafana, Prometheus, Kubernetes.
%   \item Details:
%     \docs[Kubebuilder Grafana Plugin]{https://book.kubebuilder.io/plugins/grafana-v1-alpha.html} \textperiodcentered\
%     \youtube[Show Case]{https://www.youtube.com/watch?v=-w\_JjcV8jXc&t=2s&ab\_channel=CamilaMacedo} \textperiodcentered\
% \end{itemize}

% Reference Test
%\datedsubsection{\textbf{Paper Title\cite{zaharia2012resilient}}}{May. 2015}
%An xxx optimized for xxx\cite{verma2015large}
%\begin{itemize}
%  \item main contribution
%\end{itemize}

% \section{\faHeartO\ Honors and Awards}
% \datedline{\textit{\nth{1} Prize}, Award on xxx }{Jun. 2013}
% \datedline{Other awards}{2015}

% \section{\faInfo\ Miscellaneous}
% \begin{itemize}[parsep=0.5ex]
%   % \item Blog: http://your.blog.me
%   \item GitHub: https://github.com/Kavinjsir
%   \item Languages: English - Fluent, Mandarin - Native speaker
% \end{itemize}

%% Reference
%\newpage
%\bibliographystyle{IEEETran}
%\bibliography{mycite}
\end{document}
